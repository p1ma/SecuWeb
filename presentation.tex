\documentclass{beamer}

\mode<presentation>
{
	\usetheme{Warsaw}
	\usecolortheme{default}
	\usefonttheme{serif}
	\setbeamertemplate{navigation symbols}{}
	\setbeamertemplate{caption}[numbered]
} 

\usepackage[francais]{babel}
\usepackage[utf8x]{inputenc}


% Presentation informations
\title[Sécurité WEB et audit]{Sécurité WEB et audit}
\author{P. Junges, F.Nosari}
\institute[F.S.T]{Faculté des Sciences et Technologies}
\date{\today}

\begin{document}
	
	\begin{frame}
		\titlepage
	\end{frame}

	% Table of contents
	\begin{frame}{Sommaire}
	\tableofcontents
	\end{frame}

\section{Introduction}

	\begin{frame}{Introduction}
	
		\begin{itemize}
		\item In these slides we show how Overleaf can be used with standard chemistry packages to easily create professional presentations.
		\item If you're new to \LaTeX{}, check out this free introductory course by Overleaf founder Dr John Lees-Miller: \url{www.overleaf.com/blog/7}
		\item You can also find more quick tips and tricks on the help pages at \url{www.overleaf.com/help}
		\end{itemize}
	
	
	
	\end{frame}

		\subsection{The chemistry packages}
			\begin{frame}{The chemistry packages}
			
				We focus on two \LaTeX{} chemistry packages:
				\begin{block}{The \texttt{chemfig} package}
				This package provides the command which draws molecules. Created by Christian Tellechea, a detailed user guide can be found here:\\[0.4cm]
				\small{\url{www.tex.ac.uk/ctan/macros/generic/chemfig/chemfig_doc_en.pdf}}
				\end{block}
			
			
			\end{frame}



\end{document}