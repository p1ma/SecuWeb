\documentclass{beamer}

\mode<presentation>
{
	\usetheme{Juanlespins}
	\usecolortheme{default}
	\usefonttheme{serif}
	\setbeamertemplate{navigation symbols}{}
	\setbeamertemplate{caption}[numbered]
} 

\usepackage[francais]{babel}
\usepackage[utf8x]{inputenc}


% Presentation informations
\title{Sécurité Web et audit}
\author{P-M. Junges, F.Nosari}
\institute{Faculté des Sciences et Technologies}
\date{\today}

\begin{document}

\begin{frame}
	\titlepage
\end{frame}

\begin{frame}{Sommaire}
	\tableofcontents
\end{frame}

\section{Introduction}
	\begin{frame}{Introduction}
		\begin{itemize}
			\item %TODO
		\end{itemize}
	\end{frame}

% Mon avis
\begin{comment}
	Ca me semble "complet" mais si on doit faire 15 mins de pres
	on risque de prendre pas mal de temps sur cette partie

	I Audit de sécurité
		Definition
			donner def + norme ISO
		Pré requis pour l'audit
			6 principes à respecter
		Preparer l’audit
			expliquer graphique de la page 5 de la norme ISO
		Execution de l’audit
			tests d'intrusion
				def boite blanche, noire, grise
				outils d'intrusion
			test de vulnerabilites
				Outils de tests de vulnerabilites
		Conclure audit
			donner recommendations sur ce qu'il faut changer etc

	En y réfléchissant on aurait moyen de faire notre presentation sur une seule partie
	qui serait donc la partie Audit de sécurité (pour un site web) et ensuite toute la partie
	faille(injection etc.) on en parlerai dans la partie Execution de l'audit.

	Voila, j'attends ton feed back gros !
\end{comment}
% /Mon avis

\section{Audit de sécurité}
	\subsection{Présentation}
	\begin{frame}{Présentation}
		\begin{itemize}
			\item Vue à un instant T de la sécurité d'un système
			\item Pratique encadré légalement
			\item Comparaison à un référentiel (loi, politique interne, références)
		\end{itemize}
	\end{frame}
	\subsection{Motivations}
	\begin{frame}{Motivations}
		\begin{itemize}
			\item Réagir à une attaque
			\item Se faire une bonne idée du niveau de sécurité du SI
			\item Tester la mise en place effective de la politique de sécurité ;
			\item Tester un nouvel équipement ou une nouvelle application
			\item Évaluer l'évolution de la sécurité (audit périodique).
		\end{itemize}
	\end{frame}
	\subsection{Pratiques}
	\begin{frame}{Pratiques}
		\begin{itemize}
			\item Interviews
			\item Tests d'intrusion
			\item Relevés de configuration
			\item Audit de code
			\item Fuzzing
		\end{itemize}
	\end{frame}
	\subsection{Outils}
	\begin{frame}{Outils}
		\begin{itemize}
			\item OS dédiés (Kali, Pentoo, Black Arch, etc)
			\item Logiciels dédiés
			\item Tests spécifiques			
		\end{itemize}
	\end{frame}
	\begin{frame}{Outils - Web}
	\begin{itemize}
		\item W3AF
		\item Metasploit
		\item OWASP Zap	
		\item SQLMap
		\item beEF XSS Framework
	\end{itemize}
	\end{frame}

\section{Open Web Application Security Project (OWASP)}
	\subsection{Présentation}
	\begin{frame}{Présentation}
		\begin{itemize}
			\item Open Web Application Security Project
			\itemp Les dix risques de sécurité des applications Web les plus critiques		
		\end{itemize}
	\end{frame}
	
	\subsection{Top 10}
	\begin{frame}{1. Injection}
		\begin{itemize}
			\item %TODO
		\end{itemize}
	\end{frame}
	\begin{frame}{2. Broken Authentication and Session Management}
		\begin{itemize}
			\item %TODO
		\end{itemize}
	\end{frame}
	\begin{frame}{3. XSS (Cross site scripting)}
		\begin{itemize}
			\item %TODO
		\end{itemize}
	\end{frame}
	\begin{frame}{4. Broken Access Control}
		\begin{itemize}
			\item %TODO
		\end{itemize}
	\end{frame}
	\begin{frame}{5. Security Misconfiguration}
		\begin{itemize}
		\item %TODO
		\end{itemize}
	\end{frame}
	\begin{frame}{6. Sensitive Data Exposure}
		\begin{itemize}
			\item %TODO
		\end{itemize}
	\end{frame}
	\begin{frame}{7. Insufficient Attack Protection}
		\begin{itemize}
			\item %TODO
		\end{itemize}
	\end{frame}
	\begin{frame}{8. CSRF (Cross-Site Request Forgery)}
		\begin{itemize}
			\item %TODO
		\end{itemize}
	\end{frame}
	\begin{frame}{9. Using Components with Known Vulnerabilities}
		\begin{itemize}
			\item %TODO
		\end{itemize}
	\end{frame}
	\begin{frame}{10. Underprotected APIs}
		\begin{itemize}
			\item %TODO
		\end{itemize}
	\end{frame}
	
\section{Notes pour trop tard}
	\begin{frame}{Notes pour trop tard}
		\begin{itemize}
			\item %TODO
		\end{itemize}
	\end{frame}

\end{document}