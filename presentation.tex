\documentclass{beamer}

\mode<presentation>
{
	\usetheme{Madrid}
	\usecolortheme{default}
	\usefonttheme{serif}
	\setbeamertemplate{navigation symbols}{}
	\setbeamertemplate{caption}[numbered]

\makeatother
\setbeamertemplate{footline}
{
  \leavevmode%
  \hbox{%
  \begin{beamercolorbox}[wd=.4\paperwidth,ht=2.25ex,dp=1ex,center]{author in head/foot}%
    \usebeamerfont{author in head/foot}\insertshortauthor
  \end{beamercolorbox}%
  \begin{beamercolorbox}[wd=.6\paperwidth,ht=2.25ex,dp=1ex,center]{title in head/foot}%
    \usebeamerfont{title in head/foot}\insertshorttitle\hspace*{16em}
    \insertframenumber{} / \inserttotalframenumber\hspace*{1ex}
  \end{beamercolorbox}}%
  \vskip0pt%
}
\makeatletter

}

\usepackage[french]{babel}
\usepackage[utf8x]{inputenc}
\usepackage{comment}
\newcommand{\colorized}[1]{{\color{red}{#1}}}


% Presentation informations
\title{Sécurité Web et audit}
\author{Pierre-Marie JUNGES, Florent NOSARI}
\institute[UL] {
	Université de Lorraine \\
}

\date{\today}

\begin{document}

\setbeamertemplate{headline}{}
  \begin{frame}
  	\titlepage 
  \end{frame}

\begin{frame}{Introduction}
	\begin{itemize}
		\item Audit de Sécurité
		\item +
		\item Sécurité Web
		\item = 
		\item Audit de sécurité d'une application web
	\end{itemize}
\end{frame}

\begin{frame}{Plan}
	\tableofcontents
\end{frame}


\section{Initier l'audit}

	\subsection{Termes et definitions}
	\begin{frame}{Termes et definitions}
		\begin{itemize}
			\item Vue à un instant T de la sécurité d'un système
			\item Pratique encadré légalement (accord avec l'audité)
			\item Comparaison à un référentiel (loi, politique interne, références)
			\item Action ponctuelle
		\end{itemize}
	\end{frame}


	\subsection{Principes à respecter}
	\begin{frame}{Principes à respecter}
		\begin{itemize}
			\item Intégrité
			\item Précision
			\item Professionnalisme
			\item Confidentialité
			\item Impartialité
			\item Approche factuelle
		\end{itemize}
	
	\end{frame}


	\subsection{Mise en place}
	\begin{frame}{Mise en place}
	     Contact avec l'audité (réunion) : 
		\begin{itemize}
			\item Objectif
			\item Porté
			\item Méthodes (boite blanche/grise/noire)
			\item Composition de l'équipe
			\item Demandes techniques et administratives
			\item Définition des échéances
			\item Confirmation légale
		\end{itemize}		
	\end{frame}

	\begin{frame}{Mise en place}
		Contact avec l'audité (réunion) : 
		\begin{itemize}
			\item Objectif \colorized{Garantir une application sans failles de l'OWASP Top 10}
			\item Porté \colorized{Application web}
			\item Méthodes \colorized{Boite blanche}
			\item Composition de l'équipe
			\item Demandes techniques et administratives \colorized{code source}
			\item Définition des échéances \colorized{\today}
			\item Confirmation légale
		\end{itemize}		
	\end{frame}


\section{Execution de l'audit}	
	\subsection{Différentes pratiques}
		\begin{frame}{Différentes pratiques}
			Différentes pratiques existent : 
			\begin{itemize}
				\item Audit organisationnel
				\item Tests d'intrusion (fuzzing)				
				\item Revue de code source
				\item Relevés de configuration
				\item Analyse d'architecture	
			\end{itemize}		
		\end{frame}
	
		\begin{frame}{Différentes pratiques}
			Différentes pratiques existent : 
			\begin{itemize}
				\item Audit organisationnel
				\item \colorized{Tests d'intrusion (fuzzing)}
				\item \colorized{Revue de code source}
				\item \colorized{Relevés de configuration}
				\item Analyse d'architecture
			\end{itemize}		
		\end{frame}
	
	\subsection{Tests d'intrusion} 
		\begin{frame}{Outils utilisés}
				%Images/logo			
		\end{frame}
		\begin{frame}{Outils utilisés}
			%Images/logo			
		\end{frame}




\end{document}